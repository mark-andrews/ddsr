\begin{figure}
\begin{center}
\begin{tikzpicture}[every node/.style={draw,rectangle,rounded corners, inner sep=2mm}]
\node (s1) at (0,0) {$s_1$};
\node[right=of s1] (s2) {$s_2$}; 
\node[draw=none,right=of s2] (sdots1) {$\ldots$};
\node[right=of sdots1] (s3) {$s_J$}; 
\node[right=30mm of s3] (w1) {$w_1$};
\node[right=of w1] (w2) {$w_2$}; 
\node[draw=none,right=of w2] (wdots1) {$\ldots$};
\node[right=of wdots1] (w3) {$w_K$}; 
\node (r1) at (4,-3) {$r_1$};
\node[right = of r1] (r2) {$r_2$};
\node[right = of r2] (r3) {$r_3$};
\node[draw=none,right = of r3] (rdots) {$...$};
\node[right = of rdots] (rn) {$r_n$};

\draw[->] (s1) to (r1);
\draw[->] (w1) to (r1);
\draw[->] (s3) to (r3);
\draw[->] (w1) to (r3);
\draw[->] (s2) to (r2);
\draw[->] (w2) to (r2);
\draw[->] (w3) to (rn);
\draw[->] (s3) to (rn);
\end{tikzpicture}
\end{center}
\caption{A \textit{crossed} multilevel data arrangement, such as would arise in a lexical decision experiment. Each $r_i$ is grouped under one of the $s_1, s_2 \ldots s_J$ and also one of the $w_1, w_2 \ldots w_K$.}
\label{fig:crossed_diagram}
\end{figure}