% LaTeX packages
\usepackage{xspace}
\usepackage{booktabs}
\usepackage{tikz}
\usepackage{float}
\usepackage{nth}
\usepackage{lipsum}
\usepackage{mathtools}
\usepackage{pgfplots}
\usepackage{fontawesome}

% this is needed somewhere but not everywhere
% and conflicts with subfigure
%\usepackage{subcaption}

% LaTeX newcommands etc
\def\argmax{\mathop{\rm argmax}}
\def\argmin{\mathop{\rm argmin}}
\newcommand{\aicc}{$\textsc{aic}_{c}$\xspace}
\newcommand*{\aic}{\textsc{aic}\xspace}
\newcommand{\betatrue}{\vec{\beta}_{\star}}
\newcommand{\data}{\mathcal{D}}
\newcommand{\ecdf}{\textsc{ecdf}\xspace}
\newcommand*{\elpd}{\textsc{elpd}\xspace}
\newcommand{\gams}{\textsc{gam}s\xspace}
\newcommand{\gam}{\textsc{gam}\xspace}
\newcommand*{\gui}{\textsc{gui}\xspace}
\newcommand*{\given}{\vert}
\newcommand{\hatse}{\hat{\textrm{se}}}
\newcommand{\hatsemu}{\hat{\textrm{se}}_{\mu_\iota}}
\newcommand{\hatsey}{\hat{\textrm{se}}_{y_\iota}}
\newcommand*{\hmc}{\textsc{hmc}\xspace}
\newcommand*{\hpd}{\textsc{hpd}\xspace}
\newcommand{\fmle}{f_{\textrm{mle}}}
\newcommand{\icc}{\textsc{icc}\xspace}
\newcommand*{\iid}{\textsc{iid}\xspace}
\newcommand{\iqr}{\textsc{iqr}\xspace}
\newcommand{\latex}{\LaTeX{}\xspace}
\newcommand{\loocv}{\textsc{loocv}\xspace}
\newcommand*{\looic}{\textsc{looic}\xspace}
\newcommand{\mad}{\textsc{mad}\xspace}
\newcommand*{\mcmc}{\textsc{mcmc}\xspace}
\newcommand*{\model}{\mathcal{M}}
\newcommand{\nse}{\textsc{nse}\xspace}
\newcommand{\oop}{\textsc{oop}\xspace}
\newcommand{\Pop}[1]{\mathcal{P}( #1 )}
\newcommand{\Prob}[1]{\mathrm{P}( #1 )}
\newcommand{\PROB}[1]{\mathrm{P}\left( #1 \right)}
\newcommand*{\pvalues}{\textit{p}-values\xspace}
\newcommand*{\pvalue}{\textit{p}-value\xspace}
\newcommand{\rbfs}{\textsc{rbf}s\xspace}
\newcommand{\rbf}{\textsc{rbf}\xspace}
\newcommand{\reml}{\textsc{reml}\xspace}
\newcommand{\rmsea}{\textsc{rmsea}\xspace}
\newcommand{\rss}{\textsc{rss}\xspace}
\newcommand{\sem}{\textsc{sem}\xspace}
\newcommand{\sigmatrue}{\sigma_{\star}}
\newcommand*{\thetahat}{\hat{\theta}}
\newcommand*{\thetastar}{\theta_{\star}}
\newcommand*{\thetatrue}{\thetastar}
\newcommand*{\tpi}{\pi}
\newcommand*{\tthetaprop}{\tilde{\theta}_\prime}
\newcommand*{\ttheta}{\tilde{\theta}}
\newcommand{\vcs}{\textsc{vcs}\xspace}
\newcommand*{\waic}{\textsc{waic}\xspace}
\renewcommand{\linethickness}{0.05em} % Used in Chapter 7 for making LaTeX tables

\DeclarePairedDelimiter{\diagfences}{(}{)}
\newcommand{\diag}{\operatorname{diag}\diagfences}

% Not sure why I need this
% But I get a warning recommending that I add it.
\pgfplotsset{compat=1.16}

% TikZ settings
\usetikzlibrary{positioning,shapes,trees,arrows,shadows,arrows.meta,backgrounds,fit,matrix}

\tikzset{
    %Define style for boxes
    punkt/.style={
           rectangle,
           rounded corners,
           draw=black, very thick,
           text width=6.5em,
           minimum height=2em,
           text centered},
}

\tikzset{
  treenode/.style = {shape=rectangle,
                     rounded corners,
                     draw, 
                     align=center,
                     top color=white, 
                     font=\ttfamily\normalsize},
  conditional/.style = {treenode, bottom color = red!20},
  code/.style = {treenode, bottom color = gray!20},
  dummy/.style = {treenode},
  edge_label/.style = {font=\ttfamily\normalsize},
  every picture/.style = {level distance = 8em,
                          sibling distance=10em,
                          level 1/.style = {sibling distance=15em},
                          level 2/.style = {sibling distance=8em},
                          edge from parent/.style = {draw, -latex},
                          sloped}
}

\tikzstyle{background}=[rectangle, fill=none,
						draw=black,
                                                inner sep=0.3cm,
                                                rounded corners=3mm]

\tikzstyle{observation}=[circle,font=\small,minimum size=5mm,inner sep=0mm,
                                    draw=black!70,
                                    fill=black!10]

\tikzstyle{state}=[circle,font=\small,minimum size=5mm,inner sep=0mm,
                                   draw=black!70,
                                    fill=none]

\tikzstyle{limit}=[rectangle,font=\small,minimum size=0mm,inner sep=0mm,
                                    fill=none]

\tikzstyle{parameter}=[circle,font=\small,minimum size=5mm,inner sep=0mm,
                                   draw=black!70,
                                    fill=none]



% LaTeX float defaults
% ---------------------------
% See p.105 of "TeX Unbound" for suggested values.
% See pp. 199-200 of Lamport's "LaTeX" book for details.
%   General parameters, for ALL pages:
\renewcommand{\topfraction}{0.9}	% max fraction of floats at top
\renewcommand{\bottomfraction}{0.8}	% max fraction of floats at bottom
%   Parameters for TEXT pages (not float pages):
\setcounter{topnumber}{2}
\setcounter{bottomnumber}{2}
\setcounter{totalnumber}{4}     % 2 may work better
\setcounter{dbltopnumber}{2}    % for 2-column pages
\renewcommand{\dbltopfraction}{0.9}	% fit big float above 2-col. text
\renewcommand{\textfraction}{0.07}	% allow minimal text w. figs
%   Parameters for FLOAT pages (not text pages):
\renewcommand{\floatpagefraction}{0.7}	% require fuller float pages
% N.B.: floatpagefraction MUST be less than topfraction !!
\renewcommand{\dblfloatpagefraction}{0.7}	% require fuller float pages
